\documentclass{report}

\usepackage{amsmath}
\usepackage{enumitem}  % enumerate

\begin{document}

\chapter{Introduction to algorithm design}

n/a

\chapter{Algorithm analysis}

\section*{Notes}
The dominance pecking order:
\begin{gather*}
	n! \gg c^n \gg n^3 \gg n^2 \gg n^{1+\epsilon} \gg n\log n \gg n \gg \sqrt{n} \gg \\
	\log^2 n \gg \log n \gg \log n/\log\log n \gg \log\log n \gg \alpha(n) \gg 1
\end{gather*}

\section*{Solutions}
\paragraph{2-10}
\begin{enumerate}[label=(\alph*)]
	\item $f(n) = (n^2 - n)/2,\ g(n) = 6n.$
	
		Is $f(n) = O(g(n))$? If so, there is $c$ such that $f(n) \le cg(n)$
		for sufficiently large $n$.
		\begin{equation*}
			\frac{1}{2}\left(n^2 - n\right) \le 6n
				\ \rightarrow\ n^2 - n \le 12n
				\ \rightarrow\ n(n-1) \le 12n
		\end{equation*}
		Suppose there is such a $c$, then
		\begin{equation*}
			n(n-1) \le 12cn\ \rightarrow\ n-1 \le 12c
		\end{equation*}
		Clearly we can always find $n$ such that this inequality won't hold,
		so $f(n) \ne O(g(n))$.
		
		Is $g(n) = O(f(n))$? If so, there is $c$ such that $g(n) \le cf(n)$
		for sufficiently large $n$.
		\begin{equation*}
			6n \le \frac{1}{2}\left(n^2 - n\right)
			\ \rightarrow\ 12n \le n^2 - n = n(n-1)
			\ \rightarrow\ 12 \le n - 1
			\ \rightarrow\ 13 \le n.
		\end{equation*}
		So with $c = 1$ the inequality will hold for $n_0 \ge 13$, and $g(n) = O(f(n))$.
		
	\item $f(n) = n + 2\sqrt{n},\ g(n) = n^2.$
	
		$f(n) = O(g(n)) \Leftrightarrow f(n) \le cg(n)$ for sufficiently large $n$.
		\begin{gather*}
			n + 2\sqrt{n} \le cn^2,\ \text{with}\ c = 1, \\
			n + 2\sqrt{n} \le 2n\ \text{for $n > 4$}, \\
			2n \le n^2\ \text{so}\ f(n) = O(g(n)).
		\end{gather*}
		
		$g(n) = O(f(n)) \Leftrightarrow g(n) \le cf(n)$ for sufficiently large $n$. But
		this asks to find $c$ such that $n^2 \le c\left(n + 2\sqrt{n}\right)$; since
		ultimately $n^2 \gg n$, $g(n) \ne O(f(n))$.
		
	\item $f(n) = n\log n,\ g(n) = n\sqrt{n}.$
		\begin{gather*}
			f(n) = O(g(n)) \Leftrightarrow n\log n \le cn\sqrt{n},\ \text{with $c=1,$} \\ 
			\rightarrow\ \log n \le \sqrt{n/2},
		\end{gather*}
		since $\sqrt{n} \gg \log n,\ f(n) = O(g(n)).$

		By the same argument, $g(n) \ne O(f(n)).$
		
	\item $f(n) = n + \log n,\ g(n) = \sqrt{n}\ \rightarrow\ n + \log n \le c\sqrt{n}$, and
		since $n \gg \sqrt{n}$, any constant factor will be dominated by the linear term, so
		$f(n) \ne O(g(n)).$ Conversely and by the same argument, $g(n) = O(f(n)).$
		
	\item $f(n) = 2\left(\log n\right)^2,\ g(n) = \log n + 1.$ Note that
		$2\left(\log n\right)^2 = 2\log^2 n$, and $\log^2 n \gg \log n,$ so $g(n) = O(f(n))$
		and $f(n) \ne O(g(n)).$
		
	\item $f(n) = 4n\log n + n,\ g(n) = \left(n^2 - n\right)/2.$ We know that
		$n\log n \gg n,$ so we can consider just this term from $f(n).$ But ultimately the
		quadratic term in $g(n)$ dominates so $f(n) = O(g(n)).$
\end{enumerate}


\paragraph{2-11}
\begin{enumerate}[label=(\alph*)]
	\item HOLA
\end{enumerate}

\end{document}