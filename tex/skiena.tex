\documentclass{report}

\usepackage{amsmath}
\usepackage{enumitem}  % enumerate

\begin{document}

\chapter{Introduction to algorithm design}

n/a

\chapter{Algorithm analysis}

\paragraph{2-10}
\begin{enumerate}[label=\alph*)]
	\item $f(n) = (n^2 - n)/2,\ g(n) = 6n.$
	
		Is $f(n) = O(g(n))$? If so, there is $c$ such that $f(n) \le cg(n)$
		for sufficiently large $n$.
		\begin{equation*}
			\frac{1}{2}\left(n^2 - n\right) \le 6n
				\ \rightarrow\ n^2 - n \le 12n
				\ \rightarrow\ n(n-1) \le 12n
		\end{equation*}
		Suppose there is such a $c$, then
		\begin{equation*}
			n(n-1) \le 12cn\ \rightarrow\ n-1 \le 12c
		\end{equation*}
		Clearly we can always find $n$ such that this inequality won't hold,
		so $f(n) \ne O(g(n))$.
		
		Is $g(n) = O(f(n))$? If so, there is $c$ such that $g(n) \le cf(n)$
		for sufficiently large $n$.
		\begin{equation*}
			6n \le \frac{1}{2}\left(n^2 - n\right)
			\ \rightarrow\ 12n \le n^2 - n = n(n-1)
			\ \rightarrow\ 12 \le n - 1
			\ \rightarrow\ 13 \le n.
		\end{equation*}
		So with $c = 1$ the inequality will hold for $n_0 \ge 13$, and $g(n) = O(f(n))$.
		
	\item $f(n) = n + 2\sqrt{n},\ g(n) = n^2.$
	
		$f(n) = O(g(n)) \Leftrightarrow f(n) \le cg(n)$ for sufficiently large $n$.
		\begin{gather*}
			n + s\sqrt{n} \le cn^2,\ \text{with}\ c = 1, n = 4, \\
			4 + 2\sqrt{4} = 8 \le 4^2 = 16\ \text{so}\ f(n) = O(g(n)).
		\end{gather*}
\end{enumerate}

\end{document}